\chapter{Implementación del sistema IoT}\label{cap: }

\addcontentsline{toc}{section}{Introducción}
        \textbf{\Large Introducción}\newline
        
        En este capítulo se comentan los pasos seguidos para la implementación del sistema: se realiza la descripción del proceso a monitorear o planta, los sensores que incorpora...

\section{Descripción del sistema}

    ...


\section{Sensores} \label{sec: sensores}

    La secuencia de sensores pertenecientes al sistema es selecta, puesto que se tomaron los sensores teniendo en cuenta varios factores:

    \begin{itemize}
        \item Variable a medir
        \item Rango de operación
        \item Precisión
        \item Voltaje de alimentación
        \item Corriente de alimentación
        \item Precio
    \end{itemize}

    \vspace{1cm}

    Resumiendo en la siguiente tabla los sensores a emplear según la variable a medir.

    \begin{table}[H]
        \centering
        \caption{Relación sensores}
        \label{tab:relacion_sensores}
        \begin{tabular}{|l|l|l|}
        \hline
        \cellcolor[HTML]{9698ED}                      & \cellcolor[HTML]{9698ED}                           & \cellcolor[HTML]{9698ED}                         \\
        \multirow{-2}{*}{\cellcolor[HTML]{9698ED}No.} & \multirow{-2}{*}{\cellcolor[HTML]{9698ED}Variable} & \multirow{-2}{*}{\cellcolor[HTML]{9698ED}Sensor} \\ \hline
        1                                             & Temperatura                                        & DHT22                                            \\ \hline
        2                                             & Humedad                                            & DHT22                                            \\ \hline
        3                                             & CO2                                                & SGP30                                            \\ \hline
        4                                             & Vibración                                          & TZT-LM393                                        \\ \hline
        5                                             & Calidad de Aire                                    & ZP07-MP503                                       \\ \hline
        6                                             & Intensidad luminosa                                & BH1750                                           \\ \hline
        \end{tabular}
    \end{table}

% Se tomaron precios de referencia de los sensores de las páginas oficiales de compras online Amazon y Aliexpress para tener una idea del coste de montaje de los nodos.\\

% \begin{table}[H]
%     \centering
%     \caption{Relación precios de referencia}
%     \subcaption*{Fuente: Elavoración propia}
%     \begin{tabular}{|c|c|c|}
%     \hline
%     \textbf{Sensor} & \textbf{Precio Amazon (U)} & \textbf{Precio Aliexpress (U)} \\ \hline
%     DHT22           & 13.69 – 16.99 \$           & 2 – 4 \$                       \\ \hline
%     MH-Z19C         & -                          & 16.99 \$                       \\ \hline
%     M0168           & -                          & 0.31 – 1.34 \$                 \\ \hline
%     SDS011          & 14.62 \$                   & 17.64 \$                       \\ \hline
%     LDR             & 14.50 \$                   & 0.82 \$                        \\ \hline
%     BH1750          & -                          & 8.46 \$                        \\ \hline
%     \end{tabular}
%     \label{tab: precios_referencia}
% \end{table}

\subsection{Sensor de temperatura DHT22}

  \begin{figure}[H]
      \centering
      \includegraphics[width=5cm, height=5cm]{imagenes/dht22.jpg}
      \caption{Sensor DHT22}
      \subcaption*{Fuente: Datasheet fabricante}
      \label{imag:dht22}
   \end{figure}
   
El DHT22 (AM2302) es un sensor digital de temperatura y humedad relativa de buen rendimiento y de bajo costo. Integra un sensor capacitivo de humedad y un termistor para medir el aire circundante, y muestra los datos mediante una señal digital en el pin de datos (no posee salida analógica).\\

\begin{figure}[H]
    \centering
    \includegraphics[width=8cm, height=6cm]{imagenes/dht22 dimensiones.jpg}
    \caption{Dimensiones sensor DHT22}
    \subcaption*{Fuente: Datasheet fabricante}
    \label{imag:dimensiones_dht22}
\end{figure}

\vspace{1cm}

\subsection{Sensor de CO2 SGP30}

\begin{figure}[H]
      \centering
      \includegraphics[width=5cm, height=3cm]{imagenes/mh-z19c.png}
      \caption{Sensor MH-Z19C}
      \label{imag:mh-z19c}
   \end{figure}

El sensor de gas de dióxido de carbono MH-Z19C es un pequeño sensor inteligente de uso general que utiliza el principio del infrarrojo no disperso (NDIR) para detectar la presencia de CO2 en el aire.\\

\textbf{Otros datos}
\begin{itemize}
    \item Señal de salida: UART(TTL)
    \item Tiempo de precalentamiento: 60 segundos
    \item Temperatura de operación: De -10 a 50°C
    \item Humedad de operación: De 0 - 95 por ciento RH
    \item Dimensiones: aprox. 39 x 20 x 9 mm
    \item Tipo de conector: JST ZH de 7 pines
\end{itemize}

\vspace{0.5cm}

\subsection{Sensor de vibración M0168}

\begin{figure}[H]
      \centering
      \includegraphics[width=6.5cm, height=5cm]{imagenes/sensor-piezoelectrico.jpg}
      \caption{Sensor piezoeléctrico M0168}
      \label{imag:M0168}
   \end{figure}

En este sensor piezoeléctrico cuando el choque de la cerámica con la lámina metálica genera una señal eléctrica, esta señal analógica es la recibida por los pines analógicos de microcontroladores.\\

\textbf{Especificaciones Técnicas}

\begin{itemize}
    \item Voltaje de trabajo: 3.3V o 5V
    \item Corriente de trabajo: 1mA
    \item Rango de temperatura de funcionamiento: -10 ~ +70
    \item Interfaz Tipo: salida analógica
    \item Tamaño del artículo: 30mm x 23mm
\end{itemize}

\newpage

\subsection{Sensor de calidad de aire}

\vspace{1cm}

\begin{figure}[H]
      \centering
      \includegraphics[width=4.5cm, height=4.5cm]{imagenes/Sensor SDS011.jpg}
      \caption{Sensor SDS011}
      \label{imag:SDS011}
   \end{figure}

Se basa en el principio de dispersión láser: se puede inducir la dispersión de la luz cuando las partículas atraviesan el área de detección. La luz dispersa se transforma en señales eléctricas, después estas señales serán amplificadas y procesadas. El número y el diámetro de las partículas se pueden obtener mediante análisis porque la forma de onda tiene ciertas relaciones con el diámetro de las partículas.\\

\textbf{Otros datos}

\begin{itemize}
    \item Corriente del sueño: 2mA
    \item Frecuencia de muestreo serie: 1 segundo
    \item Resolución diámetro de partículas: <= 0.3um
    \item Rango de temperatura: -20 a 50°C
    \item Tamaño físico: 71mm x 70mm x 23mm 
\end{itemize}

\vspace{5cm}

\subsection{Sensor LDR}

\begin{figure}[H]
      \centering
      \includegraphics[width=5.5cm, height=4cm]{imagenes/sensor LDR.png}
      \caption{Sensor LDR}
      \label{imag:LDR}
   \end{figure}

Una fotorresistencia es un componente electrónico cuya resistencia disminuye con el aumento de la intensidad de la luz incidente. Puede también ser llamado fotorresistor, fotoconductor, célula fotoeléctrica o resistor dependiente de la luz, cuyas siglas, LDR, se originan de su nombre en inglés light-dependent resistor.\\

\begin{figure}[H]
    \centering
    \includegraphics[width=4.5cm, height=4cm]{imagenes/ldr.jpg}
    \caption{Sensor LDR}
    \subcaption*{Fuente: Datasheet fabricante}
    \label{imag:dimensiones LDR}
 \end{figure}

\vspace{0.5cm}

\subsection{Sensor de luz BH1750}

El Módulo BH1750 es un sensor de iluminación digital para medición de flujo luminoso (iluminancia) de la empresa Rohm Semiconductor. Componente que posee dentro de su arquitectura interna, un conversor análogo digital (ADC) de 16 bits con una salida digital de formato I2C, que facilita la integración con microcontroladores o sistemas embebidos diversos. Este módulo entrega la intensidad luminosa directamente en unidades de Lux que es equivalente a Lumen/m2.\\

\begin{figure}[H]
    \centering
    \includegraphics[width=5cm, height=4cm]{imagenes/Sensor BH1750.jpg}
    \caption{Sensor BH1750}
    \label{imag:BH1750}
 \end{figure}

\textbf{Otros datos}

\begin{itemize}
    \item Interfaz Digital: I2C
    \item Frecuencia máxima de transmisión: 400kHZ
    \item Temperatura de operación: Desde -40°C hasta 85°C
\end{itemize}

Para su correcto funcionamiento, este sensor debe ir acompañado de una serie de componentes electrónicos para su acondicionamiento.
En la figura \ref{imag:acondicionamiento_BH1750} se puede observar el acondicionamiento brindado por el fabricante.

\begin{figure}[H]
    \centering
    \includegraphics[width=11cm, height=7cm]{imagenes/acondicionamientos sensor BH1750.jpg}
    \caption{Acondicionamiento sensor BH1750}
    \subcaption*{Fuente: Datasheet fabricante}
    \label{imag:acondicionamiento_BH1750}
\end{figure}

\vspace{1cm}

\subsection{Alimentación}

\textbf{Regulador LDO RT9013.}\newline

El RT9013 es un regulador LDO de 500 mA de alto rendimiento que ofrece PSRR extremadamente alto y caída ultrabaja. Ideal para aplicaciones inalámbricas y de RF portátiles con requisitos exigentes de rendimiento y espacio.\\

La corriente de reposo RT9013 es tan baja como 25uA, lo que prolonga aún más la vida útil de la batería. El RT9013 también funciona con condensadores cerámicos de baja ESR, lo que reduce la cantidad de espacio de placa necesario para las aplicaciones de energía, lo que es fundamental en los dispositivos inalámbricos de mano.\\

El RT9013 consume 0.7uA típicos en modo de apagado y tiene un tiempo de encendido rápido de menos de 40us. Las otras características incluyen voltaje de caída ultrabajo, alta precisión de salida, protección de limitación de corriente y alta relación de rechazo de ondulación. Disponible en el paquete SC-82, SOT-23-5, SC-70-5 y WDFN-6L 2x2.\\

\textbf{Características}

\begin{itemize}
    \item Amplios rangos de voltaje de operación: 2.2V a 5.5V
    \item Caída baja: 250mV a 500mA
    \item Ruido ultrabajo para aplicaciones de RF
    \item Respuesta ultrarrápida en transitorios de línea/carga
    \item Protección de limitación de corriente
    \item Protección de apagado térmico
    \item Tasa de rechazo de fuente de alimentación alta
    \item A la salida solo se requiere 1 uF de condensador para la estabilidad
    \item Entrada de apagado controlado por lógica TTL
\end{itemize}

\vspace{1cm}

\begin{figure}[H]
    \centering
    \includegraphics[width=5.5cm, height=6cm]{imagenes/esquematico RT9013.pdf}
    \caption{Configuración de pines}
    \subcaption*{Fuente: Datasheet fabricante}
    \label{imag:pines_RT9013}
\end{figure}

\begin{figure}[H]
    \centering
    \includegraphics[width=12cm, height=7cm]{imagenes/acondicionamiento RT9013.jpg}
    \caption{Acondicionamiento}
    \subcaption*{Fuente: Datasheet fabricante}
    \label{imag:acondicionamiento_RT9013}
\end{figure}

\newpage

\section{Condiciones medioambientales...} \label{sec: condiciones_medioambientales}


    \begin{figure}[H]
        \centering
        \includegraphics[width=16cm, height=5cm]{imagenes/gráfica_comparativa_variables_medioambientales.png}
        \caption{Correlación de valores medioambientales}
        \subcaption*{Fuente: Elaboración propia}
        \label{imag:grafica_condiciones_medioambientales}
    \end{figure}


\addcontentsline{toc}{section}{Conclusiones}
        \textbf{\Large Conclusiones}\newline
