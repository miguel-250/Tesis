\begin{center}
    \section*{Abstract}    
\end{center}

This Diploma Work deals with the process of designing and installing a monitoring and visualization system of variables to control the state of conservation of the pieces belonging to the exhibition located at the Francisco Prat Puig Cultural Center. Generalities about the application of the IoT in museums at an international level and in Cuba are discussed. A characterization of the sensors to be used is carried out according to the variables present, as well as the assembly of a security system.\\
In accordance with the characteristics of the place, the assembly of said system is carried out, facilitating the control of the state of the pieces and providing the user with the necessary data for decision-making in the event of future deterioration events for the long-term conservation of said sample.\\
This system can be used in subsequent projects that relate to the control of existing variables in a site, facilitating the supervision of objects and early warning in the event of possible effects.