\chapter{Análisis Teórico de la Aplicación del Internet de las Cosas en los Museos} \label{cap:analisis_teorico}

\addcontentsline{toc}{section}{Introducción}
        \textbf{\Large Introducción}\newline

           
    % \section{Patrimonio Cultural}

    % Hay diversas perspectivas desde las que se puede afrontar una reflexión sobre el patrimonio. Podría comenzarse, por ejemplo, con un análisis formal del concepto; un enfoque que atienda no solo a su sentido etimológico o semántico, sino también al modo en que ha sido tratado en la literatura antropológica reciente. En este sentido, hablar de patrimonio es hablar de lo que se posee, de la hacienda y bienes -espirituales o materiales, muebles o inmuebles- de una persona, de una familia o de un grupo. De este concepto general se pueden seguir otras nociones derivadas, tales como patrimonio económico, patrimonio histórico o patrimonio cultural.

    % El patrimonio cultural es la suma de todos los bienes culturales acumulados voluntariamente por una comunidad dada. Un bien cultural es determinado como tal, solo cuando la comunidad lo selecciona como elemento que debe ser conservado por poseer valores que trascienden su uso o función primitiva \cite{patrimonioCultural_Identidad}. Por su parte la Organización de las Naciones Unidas para la Educación, la Ciencia y la Cultura (UNESCO) define que lo comprenden las obras de sus artistas, arquitectos, músicos, escritores y sabios, así como las creaciones anónimas, surgidas del alma popular, y el conjunto de valores que dan sentido a la vida, es decir, las obras materiales y no materiales que expresan la creatividad de ese pueblo; la lengua, los ritos, las creencias, los lugares y monumentos históricos, la literatura, las obras de arte y los archivos y bibliotecas.

    % El concepto que se define en Cuba en el Decreto No. 118 de la Ley No. 1, Ley de Protección al Patrimonio Cultural (1983), establece en su artículo 1 que: El patrimonio cultural de la nación está integrado por aquellos bienes, muebles e inmuebles, que son la expresión o el testimonio de la creación humana o de la evolución de la naturaleza y que tienen especial relevancia en relación con la arqueología, la prehistoria, la historia, la literatura, la educación, el arte, la ciencia y la cultura en general.
 
    \section{Internet de las Cosas}

    El IoT como arquitectura emergente basada en la Internet global presta la posibilidad de intercambio de bienes y servicios entre redes de la cadena de suministro y que tiene un impacto importante en la seguridad y privacidad de los actores involucrados \cite{weber}. 

    \subsection{IoT en los museos internacionales}\label{sec:iotMundo}
        
        En el grupo de las bibliografías consultadas existen diversos ejemplos del empleo del IoT en los museos.

        En el caso del Conjunto Monumental de San Domenico Maggiore, ubicado en Nápoles (Italia), dentro del mismo se han transformado más de 270 esculturas en obras de arte parlantes. Equipado con un tablero de sensores, cada objeto puede proponerse automáticamente a los visitantes, compartiendo su historia en diferentes modalidades e idiomas, lo que permite un proceso de disfrute novedoso durante una experiencia cultural. \cite{monumentoSanDomenico}.
    
        En la Universidad de El Cairo, Egipto, desarrollaron un sistema para la conservación de las piezas en los museos, un sistema que no solo mide los atributos del entorno, sino que también mantiene la seguridad de los artefactos al detectar cualquier prueba de contacto o movimiento. También controla la intensidad de la luz en función de la ocupación de la sección del museo. Una característica diferenciadora del sistema es el diseño de energía ultrabaja de su nodo sensor que conduce a una larga vida útil de hasta 50 días. \cite{ultraLowPowerConservacion}.\\
    
        Estos son solo dos de los ejemplos. La tabla 1.2 muestra una relación de otros autores que emplean el IoT en los museos. Pero antes, para poder comprender la tabla 1.2 debemos definir una primera tabla en la que se le otorga a cada autor un número de manera tal que se entienda quien es cada autor. Por lo que se define la tabla 1.1:

    \begin{table}[H]
        \centering
        \caption{Numeración autores}
        \begin{tabular}{|c|c|}
        \hline
        \rowcolor[HTML]{9698ED} 
        Autor & Numeración Asociada \\ \hline
        Maksimovic y Cosovic \cite{maksimovic}     & 1                   \\ \hline
        Shah y Mishra \cite{shan}     & 2                   \\ \hline
        Marshall \cite{marshall}     & 3                   \\ \hline
        Ghosh, Roy, y Saha \cite{ghosh}     & 4                   \\ \hline
        Rao, Sharma, y Narayan \cite{rao}     & 5                   \\ \hline
        Alletto y cols. \cite{alletto}     & 6                   \\ \hline
        Alsuhly y Khattab \cite{alsuhly}     & 7                   \\ \hline
        Spachos y Plataniotis \cite{spachos}     & 8                   \\ \hline
        (Lopez-Martínez, Iglesias, y Carrera \cite{lopezmartinez}     & 9                   \\ \hline
        \end{tabular}
        \label{tab:numeracion_autores}
    \end{table}

    \begin{table}[H]
        \centering
        \caption{IoT en los museos internacionales}
        \begin{tabular}{|l|l|l|}
        \hline
        \rowcolor[HTML]{9698ED} 
        Manifestación       & Mediciones                                                                                                                                                     & Tecnología                                                                                                                                              \\ \hline
        Temperatura         & {(}1,2,7{)} Temperatura °C                                                                                                                                     & \begin{tabular}[c]{@{}l@{}}{(}1{)} RPI, Sensores, Cloud\\ SVM (Árboles de Decisiones),\\ {(}2{)} IoT-WSMP,\\ {(}7{)} RPI, ESP32, Node-Red.\end{tabular} \\ \hline
        Humedad             & {(}1,2,7{)} Humedad \%                                                                                                                                         & \begin{tabular}[c]{@{}l@{}}{(}1{)} RPI, Sensores, Cloud\\ SVM (Árboles de Decisiones),\\ {(}2{)} IoT-WSMP,\\ {(}7{)} RPI, ESP32, Node-Red.\end{tabular} \\ \hline
        Intensidad luminosa & {(}2,7{)} Lumens                                                                                                                                               & \begin{tabular}[c]{@{}l@{}}{(}2{)} IoT-WSMP,\\ {(}7{)} RPI, ESP32, Sensores.\end{tabular}                                                               \\ \hline
        Vibraciones         & {(}1{)} Estabilidad del Edificio (Hz)                                                                                                                          & \begin{tabular}[c]{@{}l@{}}{(}1{)} RPI, Sensores, Cloud\\ SVM (Árboles de Decisiones).\end{tabular}                                                     \\ \hline
        Humo                & {(}1{)} C02 ppm                                                                                                                                                & \begin{tabular}[c]{@{}l@{}}{(}1{)} RPI, Sensores, Cloud\\ SVM (Árboles de Decisiones).\end{tabular}                                                     \\ \hline
        Polución            & {(}1{)} Polvo                                                                                                                                                  & \begin{tabular}[c]{@{}l@{}}{(}1{)} RPI, Sensores, Cloud\\ SVM (Árboles de Decisiones).\end{tabular}                                                     \\ \hline
        Xilófagos           & {(}1{)} Plagas de madera                                                                                                                                       & \begin{tabular}[c]{@{}l@{}}{(}1{)} RPI, Sensores, Cloud\\ SVM (Árboles de Decisiones).\end{tabular}                                                     \\ \hline
        Personas            & \begin{tabular}[c]{@{}l@{}}{(}7{)} Aceleracion, Toque.\\ {(}8{)} Geolocalización.\\ {(}3{)} Movimiento\\ {(}9{)} Propuestas de juegos en el museo\end{tabular} & \begin{tabular}[c]{@{}l@{}}{(}1{)} PIR, {(}4{)} IA,\\ {(}5{)} RPI, ESP32, Sensores.\end{tabular}                                                        \\ \hline
        \end{tabular}
        \label{tab:iot_internacional}
    \end{table}

    \section{Internet de las Cosas en Cuba}\label{sec:iotCuba} 

    En Cuba existen escasos ejemplos de la aplicación del Internet de las Cosas en los museos, dentro de ellos se encuentra el Palacio del Segundo Cabo.\\
        
    Desde 2012 se venía proyectando una sala dedicada al libro cubano, con la cooperación de la Unión Europea y la UNESCO. \cite{oficinaHistoriadorRescatepatrimonio}. Gracias a este Rescate Patrimonial y Desarrollo Cultural en La Habana, se propició la restauración del mismo para la creación del centro. Esta iniciativa de cooperación internacional fue desarrollada por la Oficina del Historiador de la Ciudad de La Habana, con apoyo de la UNESCO \cite{potencialidadProyectoMuseologico}.\\
        
    El centro cuenta con una sala monográfica que propone un audiovisual realizado por autores cubanos donde se relata la historia del inmueble donde está ubicada la institución. Además de una sala introductoria que refleja la visión que se tenía antes del encuentro de las dos culturas con la llegada de Cristóbal Colón. Este relata en paralelo la historia de Cuba y de Europa interrelacionadas en un ambiente inmersivo y por fuera se puede recorrer desde la visión de las artes plásticas europeas y cubanas. Siempre están acompañados de cuatro audiovisuales que recorren líneas independientes como la historia de la esclavitud, la ciencia, los servicios públicos y la economía.\\
        
    Adosada a esta sala se encuentra la de llegada y migraciones donde en un espacio se reproduce una parte de un galeón, atractivo para el público pues puede sentirse una ligera «brisa de mar», una corriente de aire situada en la parte posterior del barco. Esto se complementa con dos pantallas y proyectores que permiten la proyección de un audiovisual que también narra la llegada de los aborígenes, los conquistadores; la migración forzada de los africanos y así sucesivamente con las diferentes olas migratorias que han formado la actual característica mestiza de nuestra identidad y nación.\\
        
    La Sala de Viajeros se soporta en un software que despliega una galería de imágenes ya sea de cubanos relevantes de visita por Europa o de europeos de visita en Cuba. Mediante el uso de pantallas táctiles, el visitante puede interactuar con las imágenes de las personalidades, leer una síntesis biográfica e incluso escuchar muestras de audio.
    
        
    \section{Arquitectura IoT y sus principales capas}\label{sec:arquitecturas}

    No existe una única definición universalmente adoptada, estándar, de Arquitectura de IoT; diferentes propuestas han surgido durante su desarrollo. Se abarcan tecnologías de comunicación, dispositivos de cómputo, sensores y actuadores \cite{ioT_en_Cosas_de_salud}.
    
    La arquitectura de IoT es principalmente desarrollada por capas, dígase, la arquitectura de 3 capas, la arquitectura de 5 capas, la arquitectura de Nube, la arquitectura de niebla y la arquitectura de computación de Borde, solo por mencionar algunas \cite{arquitecturaIEEE}.\\

    \begin{figure}[H]
        \centering
        \includegraphics[width=8cm, height=5cm]{imagenes/Comparacion-arquitecturas-1024x535}
        \caption{Comparación arquitecturas}
            \subcaption*{Fuente: \cite{arquitecturaPaginaLuisGarcia}}
        \label{imag:comparacionArquitecturas}
    \end{figure}

    En la figura \ref{imag:comparacionArquitecturas} se desarrolla una comparación entre las arquitecturas de 3 capas, 5 capas y la arquitectura niebla.

    Según Ciberseguridad \cite{capasIoTciberseguridad}, la mayoría de estas arquitecturas de IoT se basan en fundamentos básicos:
    \begin{itemize}
        \item Dispositivos más inteligentes en una forma diferente.
        \item Red y puerta de enlace que permite que los dispositivos formen parte del IoT.
        \item Middleware que incluye espacios de almacenamiento de datos y avances en las capacidades de predicción.
        \item Aplicaciones de usuario final.
    \end{itemize}

    Existen varias arquitecturas, marcos de referencia o modelos conceptuales para IoT propuestos por organizaciones, comunidad académica y el sector empresarial. Las propuestas de arquitecturas pueden variar de autor en autor, en dependencia de la estructura del sistema IoT propuesto. Dichas arquitecturas son desarrolladas por capas en las que se agrupan los objetos, dispositivos, sensores, actuadores, entre otros. \cite{internetOfThingsStateOfTheArt} \cite{capasIoTciberseguridad}.\\

    En la figura \ref{imag:modelos_arquitecturas_iot} se representa una comparativa de algunos modelos basados en capas. Para propiciar una mejor comprensión de la figura \ref{imag:modelos_arquitecturas_iot} se desarrolló la tabla \ref{tab: referencias_capas} donde se muestra la relación de los autores referidos a las capas del IoT.

    \begin{table}[H]
        \centering
        \caption{Referencias figura \ref{imag:modelos_arquitecturas_iot}}
        \subcaption*{Fuente: Elaboración propia}
        \begin{tabular}{|l|l|}
        \hline
        \textbf{No. de Capas}    & \textbf{Referencias}                                                                                                                                              \\ \hline
        \multirow{2}{*}{3 capas} & \multirow{2}{*}{\begin{tabular}[c]{@{}l@{}}\cite{ref10} \cite{ref13} \cite{ref15}\\ \cite{ref16} \cite{ref19} \cite{ref25}\end{tabular}} \\
                                 &                                                                                                                                                                   \\ \hline
        4 capas                  & \begin{tabular}[c]{@{}l@{}}\cite{ref10} \cite{ref12} \cite{ref13}\\ \cite{ref19} \cite{ref16}\end{tabular}                          \\ \hline
        5 capas                  & \begin{tabular}[c]{@{}l@{}}\cite{ref9} \cite{ref10} \cite{ref13}\\ \cite{ref19} \cite{ref16}\end{tabular}                          \\ \hline
        Basado en SOA            & \begin{tabular}[c]{@{}l@{}}\cite{ref4} \cite{ref8} \cite{ref9}\\ \cite{ref15}\end{tabular}                                                        \\ \hline
        Basado en Middleware     & \cite{ref9}                                                                                                                                                \\ \hline
        6 capas                  & \cite{ref19}                                                                                                                                                        \\ \hline
        \end{tabular}
        \label{tab: referencias_capas}
    \end{table}

    \begin{figure}[H]
        \centering
        \includegraphics[width=15.6cm, height=9cm]{imagenes/capas.jpg}
        \caption{Modelos de arquitecturas IoT basados en capas}
        \subcaption*{Fuente: Elaboración propia}
        \label{imag:modelos_arquitecturas_iot}
    \end{figure}

    \newpage

    Desde el punto de vista de Ciberseguridad \cite{capasIoTciberseguridad} existen capas fundamentales dentro de la esctructura de IoT. En la tabla \ref{tab: capas_IoT} se detallan estas capas:
    \begin{table}[H]
        \centering
        \caption{Capas de IoT}
        \subcaption*{Fuente: \cite{capasIoTciberseguridad}}
        \begin{tabular}{|l|l|}
        \hline
        \rowcolor[HTML]{9698ED} 
        \textbf{Capas}                                                             & \textbf{Descripción}                                                                                                                                                         \\ \hline
                                                                                   &                                                                                                                                                                              \\
        \multirow{-2}{*}{Capa de percepción}                                       & \multirow{-2}{*}{Administra dispositivos inteligentes en todo el sistema.}                                                                                                   \\ \hline
        \begin{tabular}[c]{@{}l@{}}Capa de conectividad/\\ transporte\end{tabular} & \begin{tabular}[c]{@{}l@{}}Permite transferir datos desde la nube a los dispositivos\\ y viceversa, diferentes aspectos de las puertas de enlace\\ y las redes.\end{tabular} \\ \hline
        Capa de procesamiento                                                      & \begin{tabular}[c]{@{}l@{}}Controla y administra los niveles de IoT para optimizar\\ los datos en todo el sistema.\end{tabular}                                              \\ \hline
        Capa de aplicación                                                         & \begin{tabular}[c]{@{}l@{}}Ayuda en los procedimientos de análisis, control de\\ dispositivos e informes a los usuarios finales.\end{tabular}                                \\ \hline
        Capa empresarial                                                           & \begin{tabular}[c]{@{}l@{}}Deriva información y análisis de toma de decisiones\\ a partir de datos.\end{tabular}                                                             \\ \hline
        Capa de seguridad                                                          & \begin{tabular}[c]{@{}l@{}}Cubre todos los aspectos de protección de toda la\\ arquitectura de IoT.\end{tabular}                                                             \\ \hline
        Capa de borde                                                              & \begin{tabular}[c]{@{}l@{}}Funciona en un bode o cerca de la recopilación\\ de información del dispositivo.\end{tabular}                                                     \\ \hline
        \end{tabular}
        \label{tab: capas_IoT}
        \end{table}

    No obstante, según B. Mazon y A. Pan Olivo \cite{internetOfThingsStateOfTheArt}, dentro de las capas más importantes del IoT podemos encontrar:

    \begin{itemize}
        \item \textit{La capa de percepción} (Objetos/ Dispositivos/ SensorActuador/ WSN/ Edge Computing/ Sensado), digitaliza y transfiere datos a la capa de red, a través de canales seguros. Se localizan los objetos físicos, dispositivos sensores y actuadores utilizados para recopilar información del contexto. \cite{ref9}.
        \item \textit{La capa de acceso} (Adaptación/ Observador), comprueba la información que recibe de la capa de percepción, si está protegida o no contra intrusos y virus. Si hay algún ataque, no pasa los datos a la siguiente capa. También verifica la identidad y autenticación de los objetos. \cite{ref9} \cite{ref10}.
        \item \textit{La capa de Red} (Abstracción de Objetos/ Transporte/ Fog computing), transporta y transmite los datos, recopilados de la capa de percepción, hacia la cloud. Se localizan componentes de red (switch, router, Gateway, etc.), medios de comunicación y protocolos. También es responsable de aspectos de seguridad y el control de ataques. \cite{ref9} \cite{ref10}.
        \item \textit{La Capa Aplicación / Cloud Computing (CC)} en modelos de más de tres capas, puede dividirse en:
            \begin{itemize}
                \item \textit{Capa de procesamiento y almacenamiento, Soporte, o Middleware}. Permite a los programadores de aplicaciones IoT trabajar con objetos heterogéneos sin tener en cuenta una plataforma de hardware específica. Se encarga de integrar, almacenar, procesar y analizar datos, tomar decisiones y ofrecer servicios de protocolos de conexión de red. \cite{ref9}.
                \item \textit{La capa de aplicación}, define los servicios y funciones que proporciona la aplicación IoT implementada (hogar inteligente, ciudad inteligente, salud inteligente, etc.) a los clientes. Los servicios pueden variar para cada aplicación y depende de la información que se recopilan de los sensores. También se consideran aspectos de seguridad. \cite{ref9} \cite{ref10}.
                \item \textit{La capa empresarial}, tiene la responsabilidad de administrar y controlar el comportamiento de las aplicaciones, modelos de negocios y ganancias de IoT. También tiene la capacidad de determinar cómo se puede crear, almacenar y cambiar la información. Administra la privacidad del usuario y evita vulnerabilidades. \cite{ref9} \cite{ref10}.
            \end{itemize}
    \end{itemize} 

    \subsection{Propuesta de la Arquitectura IoT del Sistema}

    En base de los es quemas de arquitecturas propuestas, definimos el uso de la arquitectura del proyecto la cual se cimienta en la estructura de cuatro capas, considerando las diferentes capas.

    \newpage
    
    \addcontentsline{toc}{subsection}{\hspace{1.3cm}1.4.1.1\hspace{5mm}Vista funcional}
        \textbf{1.4.1.1\hspace{5mm}Vista funcional}

    En la figura \ref{imag:vista_funcional_arquitectura_iot} se observan los diferentes módulos funcionales que componen las capas de la propuesta arquitectónica. La capa de percepción se encarga de la obtención de variables asociadas a la conservación de las piezas en la muestra (sensores-microcontroladores), la capa de transporte con el propósito de transportar la data recopilada por los sensores para su posterior procesamiento y análisis, la capa de procesamiento que genera predicciones y recomendaciones expuesta a cambios constantes de los valores que se reciben desde el comienzo del ciclo en la etapa de percepción y, la capa de aplicación, la cual permite a los usuarios finales visualizar en una interfaz y mediante el empleo de la RA (Realidad Aumentada), los datos y predicciones en dependencia de la información recopilada.\\

    \begin{figure}[H]
        \hspace{5cm}
        \includegraphics[width=9cm, height=4.4cm]{imagenes/myArquitecture2.jpg}
        \caption{Vista funcional de la arquitectura}
        \subcaption*{Fuente: Elaboración propia}
        \label{imag:vista_funcional_arquitectura_iot}
    \end{figure}

    \addcontentsline{toc}{subsection}{\hspace{1.3cm}1.4.1.2\hspace{5mm}Vista de implementación}
        \textbf{1.4.1.2\hspace{5mm}Vista de implementación}

    En la vista de implementación se muestran los componentes de hardware y software del sistema propuesto, en los cuales se presenta la interfaz de usuario esperada a través del empleo de Realidad Aumentada y el circuito perteneciente a la capa de percepción
    así como la muestra del empleo de comunicaciones wifi y de RED en la capa de transporte y el concentrador RaspberryPi perteneciente a la capa de procesamiento donde se trabaja además, sobre la plataforma perteneciente al patrimonio universitario.
    En la figura \ref{imag:vista_implementacion_arquitectura_iot}, se observan las tecnologías empleadas en cada una de las capas propuestas en la arquitectura descrita.
    
    \begin{figure}[H]
        \centering
        \includegraphics[width=8cm, height=5.4cm]{imagenes/vista de implementación.jpg}
        \caption{Vista de implementación}
        \subcaption*{Fuente: Elaboración propia}
        \label{imag:vista_implementacion_arquitectura_iot}
    \end{figure}

    El alcance de este trabajo va dirigido a la explicación detallada de la capa de percepción, enfocándose en las características del microcontrolador y sensores a emplear, así como la capa de transporte donde se hace un análisis de los métodos y protocolos de comunicación empleados en el sistema.

    \subsection{Capa de percepción}\label{subsec:capa_percepcion}
    En esta capa es donde se ubican los microcontroladores encargados de comunicarse con los sensores interactuando directamente con el medio recolectando los datos.

    \begin{figure}[H]
        \centering
        \includegraphics[width=8.5cm, height=4cm]{imagenes/perception_image.jpg}
        \caption{Diagrama funcional de captura de datos}
        \subcaption*{Fuente: Elaboración propia}
        \label{imag:capa_percepcion}
    \end{figure}

    Esta sucesión de sensores y microcontrolador (figura \ref{imag:capa_percepcion}) componen todos los Nodos que se ubicarán en cada vitrina de la colección con sus sensores interconectados según sea la necesidad de cada vitrina y permitirán obtener los valores de las variables ambientales.\\

    La colección Francisco Prat Puig está distribuida de tal modo que las piezas están ubicadas en tres salas. Enumerándolas: sala 1, sala 2 y sala 3.\\
    La sala 1 (Exposición permanente) (figura \ref{imag:sala_1}) está caracterizada por la presencia de varios objetos distribuidos en vitrinas, sobre mesas, colgados, en estantes empotrados en la pared, o en pedestales. Se observan varios objetos de cerámica, marfil, metal, piedra, madera, etcétera.\newline
    
    \begin{figure}[H]
        \centering
        \includegraphics[width=8cm, height=6cm]{imagenes/sala_1.jpg}
        \caption{Sala 1}
        \label{imag:sala_1}
    \end{figure}

    En la sala 2 solamente se encuentran pinturas pertenecientes a la colección, estas poseen también condiciones de deterioro notables por el efecto de la humedad y del nivel de luz dentro del local. La humedad provoca la aparición de hongos en el lienzo lo que provoca su deterioro y la pérdida de color.\newline
    [Imagen de la sala 2]\newline
    [Imagen de uno de los cuadros donde se vea el hongo] (Nombre || Fuente).\newline
    Dentro de la sala 3 se localizan también varios objetos ubicados en pedestales mostrando colecciones de medallas y objetos personales del mismo Prat Puig y, además, vitrinas donde se hallan, principalmente, objetos de cerámica como platos, jarrones y cántaros.\newline
    [Imagen sala 3]\newline

    \

    En el gráfico que se muestra a continuación se relaciona el valor cuantitativo de las principales formas en las que se exponen las piezas dentro de estas salas:\newline

    \begin{figure}[H]
        \centering
        \includegraphics[width=10cm, height=6cm]{imagenes/formas expositivas.jpg}
        \caption{Ubicación piezas}
        \subcaption*{Fuente: Elaboración propia}
        \label{imag:ubicacion_piezas}
    \end{figure}

    Según la figura \ref{imag:ubicacion_piezas}, varios objetos están dispuestos dentro de vitrinas donde, al encontrarse protegido por paredes de cristal, se genera un microclima\footnote{microclima: según el diccionario Oxford, es el conjunto de las condiciones climáticas particulares de un lugar determinado, resultado de una modificación más o menos acusada y puntual del clima de la zona en que se encuentra influido por diferentes factores ecológicos y medioambientales.}, por lo que la condición ambiental en estas es diferente a la condición ambiental de la sala, de ahí la necesidad de incorporar nodos de tal manera que se tomen los datos dentro de estas vitrinas y, además, un nodo para la sala en general. \newline

    \addcontentsline{toc}{subsection}{\hspace{1.3cm}1.4.2.1\hspace{5mm}Variables ambiantales}
        \textbf{1.4.2.1\hspace{5mm}Variables ambiantales}

    Como resultado de los recorridos ...

    En la tabla \ref{tab:correlacion_nodos} se relaciona la variable a medir según la ubicación del nodo.

    \begin{table}[h]
        \centering
        \caption{Correlación de los nodos}
        \subcaption*{Fuente: Elaboración propia}
        \begin{tabular}{|l|c|l|c|}
        \hline
        \rowcolor[HTML]{9698ED} 
        \multicolumn{1}{|r|}{\cellcolor[HTML]{9698ED}Nodo} & \multicolumn{1}{r|}{\cellcolor[HTML]{9698ED}Ubicación}                      & Material objetos                                                                            & \multicolumn{1}{l|}{\cellcolor[HTML]{9698ED}Variable a medir}                                                                    \\ \hline
        1                                                  & \begin{tabular}[c]{@{}c@{}}General\\ Sala 1\end{tabular}                    & porcelana, barro, marfil                                                                    & \begin{tabular}[c]{@{}c@{}}temperatura, humedad,\\ polución, luminosidad,\\ CO2, vibraciones, salitre,\\ xilófagos\end{tabular}  \\ \hline
        2                                                  &                                                                             & bronce, barro, arcilla, piedra                                                              &                                                                                                                                  \\ \cline{1-1} \cline{3-3}
        3                                                  &                                                                             & colección numismática                                                                       &                                                                                                                                  \\ \cline{1-1} \cline{3-3}
        4                                                  &                                                                             & cerámica, madera, barro, bronce                                                             &                                                                                                                                  \\ \cline{1-1} \cline{3-3}
        5                                                  &                                                                             & \begin{tabular}[c]{@{}l@{}}madera, bronce, plata, cerámica,\\ porcelana, cuero\end{tabular} &                                                                                                                                  \\ \cline{1-1} \cline{3-3}
        6                                                  &                                                                             & barro, cerámica, plata                                                                      &                                                                                                                                  \\ \cline{1-1} \cline{3-3}
        7                                                  &                                                                             & bronce, marfil, madera                                                                      &                                                                                                                                  \\ \cline{1-1} \cline{3-3}
        8                                                  & \multirow{-7}{*}{\begin{tabular}[c]{@{}c@{}}Vitrinas\\ Sala 1\end{tabular}} & porcelana, barro policromado                                                                & \multirow{-7}{*}{\begin{tabular}[c]{@{}c@{}}temperatura, humedad,\\ luminosidad, vibraciones.\end{tabular}}                      \\ \hline
        9                                                  & \begin{tabular}[c]{@{}c@{}}General\\ Sala 2\end{tabular}                    & tela, madera                                                                                & \begin{tabular}[c]{@{}c@{}}temperatura, humedad,\\ polución, luminosidad,\\ CO2, salitre, xilófagos.\end{tabular}                \\ \hline
        10                                                 & \begin{tabular}[c]{@{}c@{}}General\\ Sala 3\end{tabular}                    & metal, plástico, tela, cuero, madera                                                        & \begin{tabular}[c]{@{}c@{}}temperatura, humedad,\\ polución, luminosidad,\\ CO2, vibraciones, salitre,\\ xilófagos.\end{tabular} \\ \hline
        11                                                 &                                                                             &                                                                                             &                                                                                                                                  \\ \cline{1-1}
        12                                                 &                                                                             &                                                                                             &                                                                                                                                  \\ \cline{1-1}
        13                                                 &                                                                             &                                                                                             &                                                                                                                                  \\ \cline{1-1}
        14                                                 & \multirow{-4}{*}{\begin{tabular}[c]{@{}c@{}}Vitrinas\\ Sala 3\end{tabular}} & \multirow{-4}{*}{cerámica (barro), piedra}                                                          & \multirow{-4}{*}{\begin{tabular}[c]{@{}c@{}}temperatura, humedad,\\ luminosidad, vibraciones.\end{tabular}}                      \\ \hline
        \end{tabular}
        \label{tab:correlacion_nodos}
    \end{table}

    \newpage

    \addcontentsline{toc}{subsection}{\hspace{1.3cm}1.3.3.1\hspace{5mm}Nodos}
        \textbf{1.3.3.1\hspace{5mm}Nodos}

    Los nodos que, de manera general, recopilarán los datos de cada sala, estarán compuestos por los sensores que permitan medir los valores de temperatura, humedad, polución, CO2, vibraciones y luminosidad mientras que, en el caso de los nodos dentro de las vitrinas, se tiene en cuenta las condiciones del microclima y el material de los objetos.\\
    ...

    \subsection{Capa de transporte}\label{subsec:capa_transporte}

    Esta es la capa responsable del traslado de la data a través de los demás componentes del sistema estableciendo la comunicación necesaria desde la toma de valores ambientales hasta su análisis y muestreo.    
    Según \cite{internetOfThingsStateOfTheArt}, la capa de transporte es la que se encarga de transportar y transmitir los datos, recopilados de la capa de percepción, hacia la cloud. Se localizan componentes de red (switch, router, Gateway, etc.), medios de comunicación y protocolos. También es responsable de aspectos de seguridad y el control de ataques.
    
    Los datos se promueven a través de protocolos MQTT, HTTP y TCP/IP.\newline
    
    \addcontentsline{toc}{subsection}{\hspace{1.3cm}1.3.4.1\hspace{5mm}Protocolo MQTT}
        \textbf{1.3.4.1\hspace{5mm}Protocolo MQTT}

    MQTT son las siglas MQ Telemetry Transport, aunque en primer lugar fue conocido como Message Queing Telemetry Transport. Es un protocolo de comunicación M2M (machine-to-machine) de tipo message queue. Está basado en la pila TCP/IP como base para la comunicación. En el caso de MQTT cada conexión se mantiene abierta y se “reutiliza” en cada comunicación. Es una diferencia, por ejemplo, a una petición HTTP 1.0 donde cada transmisión se realiza a través de conexión.
    MQTT fue creado por el Dr. Andy Stanford-Clark de IBM y Arlen Nipper de Arcom (ahora Eurotech) en 1999 como un mecanismo para conectar dispositivos empleados en la industria petrolera.
    Aunque inicialmente era un formato propietario, en 2010 fue liberado y pasó a ser un estándar en 2014 según la OASIS (Organization for the Advancement of Structured Information Standards). \cite{mqtt}\\

    \newpage

    \addcontentsline{toc}{subsection}{\hspace{1.3cm}1.3.4.2\hspace{5mm}Protocolo HTTP}
        \textbf{1.3.4.2\hspace{5mm}Protocolo HTTP}

    HTTP de sus siglas en inglés: “Hypertext Transfer Protocol”, es el nombre de un protocolo el cual nos permite realizar una petición de datos y recursos, como pueden ser documentos HTML. Es la base de cualquier intercambio de datos en la Web, y un protocolo de estructura cliente-servidor, esto quiere decir que una petición de datos es iniciada por el elemento que recibirá los datos (el cliente), normalmente un navegador Web. Así, una página Web completa resulta de la unión de distintos sub-documentos recibidos, por ejemplo: un documento que especifique el estilo de maquetación de la página Web (CSS), el texto, las imágenes, videos, scrips, etcétera.

    \newpage

    \addcontentsline{toc}{subsection}{\hspace{1.3cm}1.3.4.3\hspace{5mm}Protocolo TCP/IP}
        \textbf{1.3.4.3\hspace{5mm}Protocolo TCP/IP}

    La definición de TCP/IP es la identificación del grupo de protocolos de red que hacen posible la transferencia de datos en redes, entre equipos informáticos e internet. Las siglas TCP/IP hacen referencia a este grupo de protocolos:

    \begin{itemize}
        \item TCP: Es el Protocolo de Control de Trasmisión que permite establecer una conexión y el intercambio de datos entre dos anfitriones. Este protocolo proporciona un transporte fiable de datos.
        \item IP o protocolo de internet, utiliza direcciones series de cuatro octetos con formato de punto decimal (por ejemplo 75.4.160.25). Este protocolo lleva los datos a otras máquinas de la red.
    \end{itemize}

    El modelo TCP/IP permite un intercambio de datos fiable dentro de una red, definiendo los pasos a seguir desde que se envían los datos (en paquetes) hasta que son recibidos. Para lograrlo utiliza un sistema de capas con jerarquías (se construye una capa a continuación de la anterior) que se comunican únicamente con su capa superior (a la que envía resultados) y su capa inferior (a la que solicita servicios) \cite{tcp/ip}.

    %Así, esta arquitectura pretende servir de referencia para la implementación de servicios basados en IoT en el área de conservación de las piezas en los museos o instituciones con vista al desarrollo de entornos inteligentes.\\

    %\section{Sistema de comunicación}\label{sec: sistemaComunicación}

    \section{Condiciones que favorecen el ataque de insectos xilófagos}

    En el caso del control de la aparición de los xilófagos, como daño de origen biótico, se realiza el análisis de las condiciones óptimas para la aparición de los mismos influyendo en la intensidad o severidad del ataque de estos insectos. Entre estos se pueden mencionar humedad, edad de la madera, diversidad genética, población de insectos presente y temperatura,entre otros \cite{ripa2004termitas}. 
    
    \subsection{Humedad}

    El contenido de agua de la madera constituye uno de los factores más importantes que favorecen el ataque de las especies xilófagas. Maderas con un contenido de humedad sobre el 15\% favorece las infestaciones de coleópteros xilófagos, acortando significativamente sus ciclos de vida lo que aumenta sus poblaciones y la posibilidad de reinfestaciones \cite{ripa2004termitas}, esto relacionado a que la acumulación de humedad se acentúa en construcciones con escasa ventilación, la que se debe incrementar en espacios interiores.

    ...

    En el caso del control de la aparición de los xilófagos, como daño de origen biótico, se realiza el análisis de las condiciones óptimas para la aparición de los mismos. Según \cite{monitoringMoisture} y \cite{rodriguezcodigo} el porcentaje de humedad óptimo para que crezcan los xilófagos está entre el 25 y el 55\% mientras que \cite{woodPreservation} plantea que, el rango de humedad idónea puede estar entre el 35 y el 50\%. Tomando estos porcentajes de humedad idóneos para la aparición de los xilófagos, se establece como valor máximo de humedad un 15\%.

    Valores de temperatura idóneos para la aparición de xilófagos...


    %\section{Sistema de control de población}\label{sec: sistemaPoblacion}
    %Sensores PIR\\

    \addcontentsline{toc}{section}{Conclusiones}
    \textbf{\Large Conclusiones}\newline